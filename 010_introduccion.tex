\maketitle
\section{Introducción}

https://smallpdf.com/pdf-to-word \\

Este informe, preparado por Bogados Ingeniería para la compañía minera Peñoles, se enfoca en la evaluación del comportamiento histórico de la vegetación en los humedales de altura (definidos por Pliscoff como "herbazales de altitud") en la región de Atacama de la Cordillera de los Andes, Chile. La intención es generar una línea base de imagenes satelitales detallada  y realizar un monitoreo continuo de la vegetación en esta región susceptible a los impactos de la actividad minera y el cambio climático, una creciente amenaza global.\\

En el caso de Chile, las evidencias apuntan a que el cambio climático ya está afectando los ecosistemas y la biodiversidad, modificando la dinámica de la vegetación en varias regiones.\footnote{Informe de la mesa Biodiversidad. (Diciembre 2019). Impactos del cambio climático en la biodiversidad y las funciones ecosistémicas en Chile.} La actividad minera, una fuerza importante impulsora del cambio de uso del suelo, tiene el potencial de exacerbar estos efectos. Las operaciones mineras a gran escala, según diversos estudios, pueden provocar pérdida de biodiversidad, alteración de los ecosistemas y cambios en la calidad del agua y del suelo.\\

Para mitigar estos impactos y diferenciar entre los efectos del cambio climático y los de la actividad minera, este estudio propone el uso de imágenes satelitales Landsat, que abarcan desde 1984 hasta 2022. El objetivo es establecer una línea base detallada del comportamiento histórico de la vegetación y la cobertura de nieve en la región utilizando el Índice de Vegetación de Diferencia Normalizada (NDVI) y el Índice de Nieve (NSI o NDSI).\\

Al analizar y monitorear la vegetación y las condiciones climáticas en el largo plazo, se pueden identificar y mitigar los impactos dañinos potenciales causados por la actividad minera. De esta manera, la compañía minera Peñoles puede planificar sus operaciones de manera sostenible, garantizando la conservación del ecosistema local y minimizando su huella ambiental.\\

Las actividades humanas, como la minería, junto con el cambio climático, están provocando cambios significativos en los ecosistemas. La frecuencia e intensidad crecientes de incendios, derrames, sequías y otros desastres naturales y humanos, potenciados por el cambio de uso del suelo, pueden tener múltiples impactos ecológicos y socioeconómicos.\\

El seguimiento continuo y análisis de los datos históricos permitirá a la compañía minera Peñoles, y a la sociedad en su conjunto, anticipar y mitigar los impactos negativos del cambio climático y de la actividad minera en la biodiversidad y los ecosistemas. Este esfuerzo de investigación de largo plazo y monitoreo de los ecosistemas resulta vital en el contexto del cambio climático, cuyos efectos a menudo se manifiestan en periodos de tiempo más largos.\\




\maketitle
\section{Conclusiones}
{\small
\textbf{1.	El NDVI promedio :} A lo largo de los años no presenta una tendencia significativa, con una pendiente cercana a cero y un valor p alto (0.51), indicando que no hay suficiente evidencia para rechazar la hipótesis nula de que la pendiente es cero. Esto sugiere que la vegetación en promedio no ha experimentado cambios significativos desde 1985 hasta 2021.\\

Los herbazales de altura suelen ser ecosistemas bastante resistentes que se han adaptado a condiciones extremas de altitud y clima. Si no se han observado cambios significativos en el NDVI, esto podría sugerir que, a pesar de las presiones que pudieran existir, como el cambio climático o las actividades humanas, estos herbazales han logrado mantener su integridad en términos de cobertura vegetal.\\

Sin embargo, es importante tener en cuenta que este análisis no proporciona información sobre posibles variaciones espaciales (Esto se ve mas adelante) en la vegetación a lo largo del tiempo, ni sobre otros factores que puedan estar afectando la vegetación, como los cambios en el clima o en el uso de la tierra. \\




\textbf{2. La desviación estándar del NDVI:}  Muestra una tendencia significativa al alza (pendiente = 0.0004, valor p = 0.0014). Esto sugiere que la variabilidad de la vegetación ha ido aumentando con el tiempo.\\

Analizando el contexto, un aumento en la variabilidad de la vegetación podría indicar varios fenómenos. Por un lado, podría sugerir una mayor heterogeneidad en el ecosistema, con algunas áreas ganando vegetación mientras otras la pierden. Por otro lado, también podría indicar una mayor fluctuación interanual en la cobertura vegetal, posiblemente debido a factores como las variaciones climáticas o los disturbios.\\

Si bien la cobertura vegetal promedio puede no haber cambiado significativamente, el aumento de la variabilidad sugiere que el ecosistema puede estar experimentando cambios en su composición o en su respuesta a factores ambientales. Esto podría tener implicaciones para la resiliencia del ecosistema y para la gestión de la conservación.\\


\textbf{3.	El NDVI mínimo y máximo:} El mínimo muestra una tendencia significativa a la baja (pendiente = -0.0006, valor p = 0.0007), indicando que las áreas de menor vegetación han ido disminuyendo en vegetación a lo largo del tiempo. El NDVI máximo muestra  una fuerte tendencia al alza (pendiente = 0.0032, valor p = 9.13e-06). Esto indica que las áreas de mayor vegetación han ido incrementando su vegetación a lo largo del tiempo.\\

Al considerar el contexto, estas tendencias opuestas en el NDVI mínimo y máximo podrían indicar un aumento en la heterogeneidad del ecosistema. Podría ser que algunas áreas estén perdiendo vegetación (las áreas de menor vegetación) mientras que otras (las de mayor vegetación) están ganando vegetación. Esta divergencia puede ser resultado de diferentes factores, incluyendo variaciones en las condiciones ambientales, los usos de la tierra y las presiones humanas, entre otros.\\

Estas tendencias divergentes sugieren que el ecosistema está experimentando cambios que pueden estar aumentando su heterogeneidad. Mientras que algunas áreas pueden estar prosperando, otras pueden estar sufriendo. Para la gestión de la conservación, esto puede significar que se necesiten estrategias de manejo diferenciadas para estas áreas. Sería importante realizar más investigaciones para entender mejor las causas detrás de estas tendencias y desarrollar respuestas apropiadas.\\


\textbf{5.	Respecto del NSI promedio: } Ha mostrado una tendencia al alza significativa (pendiente = 0.0019, valor p = 0.0010), lo cual sugiere un incremento en la cobertura de nieve a lo largo del tiempo.\\

Al considerar el contexto, un aumento en la cobertura de nieve podría ser resultado de varios factores. Puede ser que los inviernos se están volviendo más fríos o húmedos, o que hay menos deshielo durante las temporadas de nieve debido a cambios en las temperaturas de la primavera y el verano. Este incremento en la cobertura de nieve puede tener varias implicaciones, como el incremento de la disponibilidad de agua durante la temporada de deshielo o cambios en los patrones de vegetación y fauna.\\

Este aumento en la cobertura de nieve podría tener implicaciones significativas tanto para el ecosistema como para las actividades humanas en la región. Podría tener un impacto positivo en la disponibilidad de agua para la agricultura y otros usos, así como en la salud de los ecosistemas de montaña que dependen de la nieve. Sin embargo, también podría tener impactos negativos, como el aumento del riesgo de inundaciones o avalanchas. \\

\textbf{6.	Similar a la desviación estándar:} del NDVI, la desviación estándar del NSI también muestra una tendencia significativa al alza (pendiente = 0.00099, valor p = 6.68e-07), indicando que la variabilidad de la cobertura de nieve ha aumentado con el tiempo.\\

Al considerar el contexto, un aumento en la variabilidad de la cobertura de nieve puede indicar que los patrones de nevadas están volviéndose más irregulares o impredecibles. Esto podría ser resultado de variaciones climáticas, como fluctuaciones en las temperaturas o en los patrones de precipitación. Esta mayor variabilidad puede tener varias implicaciones, incluyendo un impacto en la disponibilidad de agua, la salud de los ecosistemas y la seguridad humana. \\

Este aumento en la variabilidad de la cobertura de nieve sugiere que los patrones de nevadas pueden estar volviéndose más impredecibles. Esto podría tener importantes implicaciones para la gestión de recursos hídricos, así como para la planificación de la seguridad y la adaptación al cambio climático.   \\

\textbf{7.	NSI mínimo y máximo:} El mínimo no muestra una tendencia significativa (pendiente = -0.0011, valor p = 0.09), mientras que el NSI máximo muestra una fuerte tendencia al alza (pendiente = 0.0043, valor p = 5.16e-07), sugiriendo que las áreas de mayor cobertura de nieve han ido incrementando su cobertura de nieve a lo largo del tiempo.\\

Al considerar el contexto, estas tendencias pueden indicar que las áreas con mayor cobertura de nieve están experimentando incrementos en la cobertura, mientras que las áreas de menor cobertura no están experimentando cambios significativos. Esto podría ser resultado de variaciones en las condiciones climáticas o topográficas, entre otros factores.\\

Estas tendencias indican que la distribución de la cobertura de nieve puede estar cambiando, lo cual podría tener varias implicaciones, desde la disponibilidad de agua hasta la salud de los ecosistemas y los riesgos asociados con la nieve. Se necesitaría más investigación para entender mejor las causas de estas tendencias y desarrollar estrategias de manejo y adaptación adecuadas.\\

\textbf{8.	En la matriz de correlación :}, destaca la alta correlación positiva entre el año y el NDVI máximo (0.68) y el NSI máximo (0.74), lo que respalda los resultados de las pruebas de tendencia. También se puede notar una correlación positiva entre la desviación estándar del NDVI y del NSI (0.63), lo que indica que cuando hay mayor variabilidad en la vegetación, también hay mayor variabilidad en la nieve.\\

Al considerar el contexto, estas correlaciones pueden reflejar los cambios ambientales que ocurren a lo largo del tiempo. Específicamente, el aumento en la vegetación máxima y la cobertura de nieve máxima pueden ser reflejo de cambios en las condiciones climáticas y ecológicas. Además, la correlación entre la variabilidad de la vegetación y la nieve puede indicar que estos dos elementos del ecosistema están interrelacionados, y que los cambios en uno pueden afectar al otro.\\

Estos resultados respaldan las conclusiones previas acerca de las tendencias en la vegetación y la cobertura de nieve a lo largo del tiempo. Sugieren que estos elementos del ecosistema están cambiando y que están interrelacionados, lo que podría tener implicaciones importantes para la gestión del medio ambiente y la adaptación al cambio climático.\\

\textbf{9.	Correlacion entre máximo de NDVI y la desviación estándar de NSI (0.79)},Esta fuerte correlación positiva indica que en los años con una mayor variabilidad en los niveles de nieve (medida por la desviación estándar de NSI), también se tiende a ver un valor máximo más alto de NDVI, que es una medida de la vegetación.\\

Esta correlación puede sugerir varias cosas. Por ejemplo, podría indicar que en los años con una mayor variabilidad en los niveles de nieve, también se ven condiciones más favorables para la vegetación en ciertos momentos o lugares, lo que da lugar a valores máximos más altos de NDVI. Es posible que estos años tengan periodos de derretimiento de la nieve que proporcionan agua adicional para la vegetación, permitiendo que alcance niveles más altos de vigor.\\

Alternativamente, también podría ser que en los años con una mayor variabilidad en los niveles de nieve, también se vean variaciones más grandes en las condiciones de la vegetación, lo que da lugar a valores máximos más altos de NDVI.\\

Es importante tener en cuenta que esta es una correlación y no necesariamente indica una relación causal. Sería necesario realizar más investigaciones para entender mejor las razones subyacentes de esta correlación y si hay un mecanismo directo que vincule la variabilidad de la nieve con los valores máximos de NDVI. También es posible que esta correlación se deba a otros factores subyacentes no medidos en este estudio que afectan tanto a la nieve como a la vegetación.\\

\textbf{10.	Tendencia del NDVI:} Se observa un leve incremento en el NDVI medio (índice de vegetación) a lo largo de los años, especialmente a partir del 2013. Sin embargo, la variabilidad anual también parece aumentar, lo que sugiere que aunque la vegetación puede estar aumentando en promedio, también está experimentando fluctuaciones más extremas.\\

Al analizar el contexto, estos cambios en la vegetación pueden ser resultado de diversos factores, como cambios en las condiciones climáticas, alteraciones en el uso del suelo o variaciones en las prácticas de manejo de la tierra. Las fluctuaciones más extremas pueden sugerir una mayor incertidumbre o inestabilidad en las condiciones de la vegetación.\\

Estos resultados sugieren que aunque la vegetación puede estar aumentando en promedio, especialmente a partir de 2013, la situación es más compleja debido a las fluctuaciones más extremas. Esto podría tener varias implicaciones para la gestión del ecosistema y la adaptación al cambio climático, ya que estas fluctuaciones pueden hacer que el sistema sea más vulnerable a perturbaciones. Por lo tanto, se recomendaría continuar monitoreando la vegetación y llevar a cabo investigaciones adicionales para entender mejor las causas de estas fluctuaciones y cómo se pueden gestionar.\\


\textbf{Conclusión General:}\\

La evaluación detallada de la dinámica de la vegetación y la cobertura de nieve a través de los índices NDVI y NSI en el periodo 1985-2021, respectivamente, revela patrones y correlaciones notables que son indicativos de cambios significativos en el ecosistema estudiado. Aunque el NDVI medio, representativo de la cobertura vegetal promedio, no ha mostrado cambios sustanciales durante este período, hay evidencia de una mayor variabilidad en la cobertura vegetal y de nieve, así como un incremento en las áreas de mayor vegetación y cobertura de nieve. Estas tendencias podrían sugerir una mayor heterogeneidad en el ecosistema, con posibles cambios en la composición de la vegetación y la distribución de la cobertura de nieve.\\

La correlación positiva entre la desviación estándar del NDVI y del NSI, y entre el NDVI máximo y la desviación estándar del NSI, implica una interrelación entre la variabilidad de la vegetación y la nieve. Esto sugiere que los cambios en uno pueden influir en el otro, potencialmente impactando la resiliencia del ecosistema y la gestión de la conservación.\\

Las áreas de menor vegetación muestran una disminución en el NDVI, mientras que las de mayor vegetación indican un incremento, reafirmando la noción de una creciente heterogeneidad en la vegetación. Por otro lado, la cobertura de nieve ha experimentado un incremento general, junto con una mayor variabilidad. La cobertura máxima de nieve también ha aumentado significativamente, mientras que las áreas de menor cobertura no han experimentado cambios significativos.\\

Estos hallazgos, en su conjunto, apuntan a un ecosistema que está cambiando de manera compleja y dinámica, probablemente como resultado de una combinación de factores como las variaciones climáticas, las alteraciones en el uso del suelo y las prácticas de manejo de la tierra. Es importante tener en cuenta que estos cambios pueden tener implicaciones significativas para la gestión de la conservación y la adaptación al cambio climático.\\


\textbf{Hipótesis:}\\
Basándose en las conclusiones descritas, una posible hipótesis para futuras investigaciones podría ser que "la creciente heterogeneidad en la cobertura de vegetación y de nieve, indicada por las tendencias observadas en el NDVI y NSI, está siendo impulsada por factores ambientales cambiantes, como las variaciones climáticas y las alteraciones en el uso del suelo. Estos cambios están provocando alteraciones en la composición de la vegetación y en la distribución de la cobertura de nieve, lo cual puede estar impactando la resiliencia del ecosistema y la gestión de la conservación". Sería beneficioso realizar más investigaciones para evaluar esta hipótesis y comprender mejor las causas subyacentes de las tendencias observadas.\\
    
}
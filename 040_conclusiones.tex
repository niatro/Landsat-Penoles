\maketitle
\section{Conclusiones}
{\small
\textbf{1.	El NDVI promedio} a lo largo de los años no presenta una tendencia significativa, con una pendiente cercana a cero y un valor p alto (0.51), indicando que no hay suficiente evidencia para rechazar la hipótesis nula de que la pendiente es cero. Esto sugiere que la vegetación en promedio no ha experimentado cambios significativos desde 1985 hasta 2021.\\

\textbf{2. la desviación estándar} del NDVI muestra una tendencia significativa al alza (pendiente = 0.0004, valor p = 0.0014). Esto sugiere que la variabilidad de la vegetación ha ido aumentando con el tiempo.\\

\textbf{3.	El NDVI mínimo} ha mostrado una tendencia significativa a la baja (pendiente = -0.0006, valor p = 0.0007), indicando que las áreas de menor vegetación han ido disminuyendo en vegetación a lo largo del tiempo.\\

\textbf{4.	Por otro lado, el NDVI máximo} muestra una fuerte tendencia al alza (pendiente = 0.0032, valor p = 9.13e-06). Esto indica que las áreas de mayor vegetación han ido incrementando su vegetación a lo largo del tiempo.\\

\textbf{5.	Al respecto del NSI}, el promedio ha mostrado una tendencia al alza significativa (pendiente = 0.0019, valor p = 0.0010), lo cual sugiere un incremento en la cobertura de nieve a lo largo del tiempo.\\

\textbf{6.	Similar a la desviación estándar} del NDVI, la desviación estándar del NSI también muestra una tendencia significativa al alza (pendiente = 0.00099, valor p = 6.68e-07), indicando que la variabilidad de la cobertura de nieve ha aumentado con el tiempo.\\

\textbf{7.	El NSI mínimo} no muestra una tendencia significativa (pendiente = -0.0011, valor p = 0.09), mientras que el NSI máximo muestra una fuerte tendencia al alza (pendiente = 0.0043, valor p = 5.16e-07), sugiriendo que las áreas de mayor cobertura de nieve han ido incrementando su cobertura de nieve a lo largo del tiempo.\\

\textbf{8.	Por último, en la matriz de correlación}, destaca la alta correlación positiva entre el año y el NDVI máximo (0.68) y el NSI máximo (0.74), lo que respalda los resultados de las pruebas de tendencia. También se puede notar una correlación positiva entre la desviación estándar del NDVI y del NSI (0.63), lo que indica que cuando hay mayor variabilidad en la vegetación, también hay mayor variabilidad en la nieve.\\

\textbf{9.	Correlacion entre máximo de NDVI y la desviación estándar de NSI (0.79)},Esta fuerte correlación positiva indica que en los años con una mayor variabilidad en los niveles de nieve (medida por la desviación estándar de NSI), también se tiende a ver un valor máximo más alto de NDVI, que es una medida de la vegetación.\\

Esta correlación puede sugerir varias cosas. Por ejemplo, podría indicar que en los años con una mayor variabilidad en los niveles de nieve, también se ven condiciones más favorables para la vegetación en ciertos momentos o lugares, lo que da lugar a valores máximos más altos de NDVI. Es posible que estos años tengan periodos de derretimiento de la nieve que proporcionan agua adicional para la vegetación, permitiendo que alcance niveles más altos de vigor.\\

Alternativamente, también podría ser que en los años con una mayor variabilidad en los niveles de nieve, también se vean variaciones más grandes en las condiciones de la vegetación, lo que da lugar a valores máximos más altos de NDVI.\\

Es importante tener en cuenta que esta es una correlación y no necesariamente indica una relación causal. Sería necesario realizar más investigaciones para entender mejor las razones subyacentes de esta correlación y si hay un mecanismo directo que vincule la variabilidad de la nieve con los valores máximos de NDVI. También es posible que esta correlación se deba a otros factores subyacentes no medidos en este estudio que afectan tanto a la nieve como a la vegetación.\\

\textbf{10.	Tendencia del NDVI:} Se observa un leve incremento en el NDVI medio (índice de vegetación) a lo largo de los años, especialmente a partir del 2013. Sin embargo, la variabilidad anual también parece aumentar, lo que sugiere que aunque la vegetación puede estar aumentando en promedio, también está experimentando fluctuaciones más extremas.\\

\textbf{11. Tendencia del NSI:} En contraste, no se aprecia una tendencia clara en el NSI medio (índice de nieve) a lo largo de los años. Sin embargo, a partir del 2007, los valores máximos de NSI parecen aumentar, lo que podría indicar años con mayor presencia de nieve.\\

\textbf{12. Desviación estándar:} Los valores de desviación estándar para NDVI y NSI están incrementando levemente, lo que sugiere que la variabilidad interanual en ambas medidas está aumentando. Esto podría estar relacionado con la variabilidad climática o con cambios en las prácticas de gestión del terreno.\\
\textbf{Rangos:} Los valores mínimo y máximo de NDVI y NSI muestran una amplia gama a lo largo de los años. Esto indica que aunque las medias pueden estar aumentando, todavía hay momentos de baja vegetación y nieve.\\



\textbf{Conclusión General:}\\

Se observa que a lo largo del período de estudio (1985 a 2021), hubo variaciones significativas tanto en el índice de vegetación normalizado (NDVI) como en el índice de nieve normalizado (NSI) en la zona de estudio, situada a 3500 mts de altura en la cordillera del norte de Chile.\\

En particular, la desviación estándar del NDVI, el valor mínimo de NDVI, el valor máximo de NDVI, la media del NSI, la desviación estándar del NSI y el valor máximo del NSI mostraron tendencias significativas a lo largo del tiempo (p < 0.05). Estos resultados indican que la variabilidad de la vegetación y la nieve ha aumentado con el tiempo, lo que podría sugerir un ecosistema en cambio y probablemente en respuesta a factores externos como el cambio climático.\\

Es importante destacar que el valor medio de NDVI no mostró una tendencia significativa, lo que podría sugerir que la productividad total de la vegetación ha permanecido relativamente constante a pesar de la mayor variabilidad observada. Por otro lado, la media del NSI sí mostró una tendencia significativa, sugiriendo un cambio en la cobertura de nieve en la región a lo largo del tiempo.\\

Además, el análisis de correlación reveló asociaciones fuertes entre varias métricas. Por ejemplo, una correlación positiva fuerte entre el valor máximo de NDVI y el año sugiere un aumento en la máxima productividad de la vegetación observada cada año. De manera similar, una fuerte correlación positiva entre la media del NSI y el año sugiere un aumento en la cobertura promedio de nieve a lo largo del tiempo.\\

    
}